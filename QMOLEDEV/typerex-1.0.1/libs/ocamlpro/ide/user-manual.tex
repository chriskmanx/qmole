%%%%%%%%%%%%%%%%%%%%%%%%%%%%%%%%%%%%%%%%%%%%%%%%%%%%%%%%%%%%%%%%%%%%%%%%%%
%                                                                        %
%                        TypeRex OCaml Studio                            %
%                                                                        %
%                 Thomas Gazagnaire, Fabrice Le Fessant                  %
%                                                                        %
%  Copyright 2011-2012 OCamlPro                                          %
%  All rights reserved.  This file is distributed under the terms of     %
%  the GNU Public License version 3.0.                                   %
%                                                                        %
%  TypeRex is distributed in the hope that it will be useful,            %
%  but WITHOUT ANY WARRANTY; without even the implied warranty of        %
%  MERCHANTABILITY or FITNESS FOR A PARTICULAR PURPOSE.  See the         %
%  GNU General Public License for more details.                          %
%                                                                        %
%%%%%%%%%%%%%%%%%%%%%%%%%%%%%%%%%%%%%%%%%%%%%%%%%%%%%%%%%%%%%%%%%%%%%%%%%%

\chapter{Generating editors bindings with {\tt ocp-ide}}

{\tt ocp-ide} can generate editors bindings for the supported editors
(currently only Emacs). It intends to provide :

\begin{itemize}
\item a consistent, typed and portable interface to the main language
  constructs and libraries used by the different editors;

\item a way to check bindings consistency (like checking that a
  combination of keys is associated to one function at at most) and to
  keep the user documentation for the supported editors in sync with
  the code.
\end{itemize}

The {\tt ocp-ide} API (whose interface is {\tt IDE.Lang}) is designed
to manipulate code fragments of the intended target language. The only
target langage currently implementing the {\tt IDE.Lang} interface is
Elisp (in {\tt emacs.ml}), so we will illustrate how to use the API
using this language. In the future, this approach can be extended to
other languages.

\section{API}

In this section, we are describing the types and functions available
in {\tt IDE.Lang}. \\

The main type is {\tt 'a IDE.Lang.t}. It is used to represent code
fragments.  A values of type {\tt 'a IDE.Lang.t} is an abstract
object, representing the code fragment which, once evaluate, returns a
value of type {\tt 'a}. Internally is it stored as a raw string (the
code itself) but this internal representation can change in the
future. \\

The API is separated into:

\begin{itemize}
\item function to create basic values in the target language;
\item function to call language specific functions (related or not to
  the editor); and
\item functions to manipulate the control flow of the generated code.
\end{itemize}

\subsection{Basic type API}

The supported basic types for the target langage are {\tt unit}, {\tt
  string}, {\tt string list}, {\tt int} and {\tt(string *
  int)}\footnote{This is intended to be a location : the {\tt string}
  is the filename and the {\tt int} is the file offset}. \\

For each of the basic types, the API has functions to create some code:
\begin{itemize}
\item using an OCaml value of corresponding type; and 
\item using an external command which returns a value of corresponding
  type.
\end{itemize}

In the flolowing examples, we show the internal representation of
values of type {\tt 'a IDE.Lang.t}, which is not available to the user
directly in the API. For example, to create code snippets manipulating
{\tt string}s, one could use {\tt IDE.Lang.string} to inject an OCaml
string into the editor language and {\tt IDE.Lang.exec\_string} to
execute a command which returns a string:

\begin{verbatim}
# let s = Emacs.string "foo";;
val s : string t = "\"foo\""

# let s = Emacs.exec_string [string "uname"; string "-a"];;
val s : strint t =
  "([..](with-temp-buffer tmpbuf
     ([..](call-process "\uname\" nil tmbuf \"-a\")
          (with-current-buffer tmpbuf (buffer-string)))))"
\end{verbatim}

\subsection{Buffers}

\subsection{Editor Bindings}

\subsection{Callbacks}
